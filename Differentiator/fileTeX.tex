\documentclass[a4paper,12pt]{article}
\usepackage[T2A]{fontenc}
\usepackage[utf8]{inputenc}
\usepackage[english,russian]{babel}
\usepackage{amsmath,amsfonts,amssymb,amsthm,mathtools}
\author{By Khromov Alexey} 
\title{Differentiator \LaTeX{}} 
\date{\today}\begin{document}
\maketitle
\newpage
Производная функции находится очевидным и нетривиальным способом:

Далее будем рассматривать призводные функции по частям, дабы облегчить себе задачу.

Давайте рассмотрим подробней эту фунцию.
\begin{equation}
\left( {0 }^ {0 }\right)' =
\end{equation}
\begin{equation}
{{0 }^ {0 }}* {\left( {{ln \left( {0 }\right) }* {0 }}+ {{0 }* {\frac{{0 }}{{0 }}}}\right) }
\end{equation}

Представим ответ в полном виде:
\begin{equation}
\left( {0 }^ {0 }\right)' =
\end{equation}
\begin{equation}
{{0 }^ {0 }}* {\left( {{ln \left( {0 }\right) }* {0 }}+ {{0 }* {\frac{{0 }}{{0 }}}}\right) }
\end{equation}

Тут слегка упростим наше выражение
\begin{equation}
\left( {0 }^ {0 }\right)' =
\end{equation}
\begin{equation}
{{0 }^ {0 }}* {\left( {{ln \left( {0 }\right) }* {0 }}+ {{0 }* {\frac{{0 }}{{0 }}}}\right) }=
\end{equation}
\begin{equation}
{{0 }^ {0 }}* {\left( {0 }+ {0 }\right) }=
\end{equation}
\begin{equation}
{{0 }^ {0 }}* {0 }=
\end{equation}
\begin{equation}
0 =
\end{equation}
\begin{equation}
0 
\end{equation}

В общем, смотри, катай и изучай :)
\end{document}
