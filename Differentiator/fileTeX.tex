\documentclass[a4paper,12pt]{article}
\usepackage[T2A]{fontenc}
\usepackage[utf8]{inputenc}
\usepackage[english,russian]{babel}
\usepackage{amsmath,amsfonts,amssymb,amsthm,mathtools}
\begin{document}

Производная функции находится очевидным и нетривиальным способом:

Давайте расмотрим подробней эту фунцию.
\begin{equation}
\left( {x }^ {x }\right)' =
\end{equation}
\begin{equation}
{{x }^ {x }}* {\left( {{ln \left( {x }\right) }* {1 }}+ {\frac{{x }}{{x }}}\right) }
\end{equation}

Тут слегка упростим наше выражение
\begin{equation}
{{x }^ {x }}* {\left( {{ln \left( {x }\right) }* {1 }}+ {\frac{{x }}{{x }}}\right) }=
\end{equation}
\begin{equation}
{{x }^ {x }}* {\left( {ln \left( {x }\right) }+ {1 }\right) }=
\end{equation}
\begin{equation}
{{x }^ {x }}* {\left( {ln \left( {x }\right) }+ {1 }\right) }
\end{equation}

Производная 2 порядка равна:

Давайте расмотрим подробней эту фунцию.
\begin{equation}
\left( {x }^ {x }\right)' =
\end{equation}
\begin{equation}
{{x }^ {x }}* {\left( {{ln \left( {x }\right) }* {1 }}+ {\frac{{x }}{{x }}}\right) }
\end{equation}

Давайте расмотрим подробней эту фунцию.
\begin{equation}
\left( ln \left( {x }\right) \right)' =
\end{equation}
\begin{equation}
\frac{{1 }}{{x }}
\end{equation}

Давайте расмотрим подробней эту фунцию.
\begin{equation}
\left( \left( {ln \left( {x }\right) }+ {1 }\right) \right)' =
\end{equation}
\begin{equation}
\left( {\frac{{1 }}{{x }}}+ {0 }\right) 
\end{equation}

Давайте расмотрим подробней эту фунцию.
\begin{equation}
\left( {{x }^ {x }}* {\left( {ln \left( {x }\right) }+ {1 }\right) }\right)' =
\end{equation}
\begin{equation}
\left( {{{{x }^ {x }}* {\left( {{ln \left( {x }\right) }* {1 }}+ {\frac{{x }}{{x }}}\right) }}* {\left( {ln \left( {x }\right) }+ {1 }\right) }}+ {{{x }^ {x }}* {\left( {\frac{{1 }}{{x }}}+ {0 }\right) }}\right) 
\end{equation}

Тут слегка упростим наше выражение
\begin{equation}
\left( {{{{x }^ {x }}* {\left( {{ln \left( {x }\right) }* {1 }}+ {\frac{{x }}{{x }}}\right) }}* {\left( {ln \left( {x }\right) }+ {1 }\right) }}+ {{{x }^ {x }}* {\left( {\frac{{1 }}{{x }}}+ {0 }\right) }}\right) =
\end{equation}
\begin{equation}
\left( {{{{x }^ {x }}* {\left( {ln \left( {x }\right) }+ {1 }\right) }}* {\left( {ln \left( {x }\right) }+ {1 }\right) }}+ {{{x }^ {x }}* {\frac{{1 }}{{x }}}}\right) =
\end{equation}
\begin{equation}
\left( {{{{x }^ {x }}* {\left( {ln \left( {x }\right) }+ {1 }\right) }}* {\left( {ln \left( {x }\right) }+ {1 }\right) }}+ {{{x }^ {x }}* {\frac{{1 }}{{x }}}}\right) 
\end{equation}

Вообщем смотри, катай и изучай :)
\end{document}
